\documentclass{aghdpl}
\usepackage[polish]{babel}
\usepackage[utf8]{inputenc}
\usepackage[T1]{fontenc}
\usepackage{enumerate}
\usepackage{listings}
\usepackage{graphicx} 
\usepackage{floatflt}
\usepackage{hyperref}
\usepackage{pdflscape}
\usepackage{dirtree}

\hypersetup{
colorlinks=false,
linkbordercolor={1 1 1}, % set to white
citebordercolor={1 1 1}, % set to white
urlbordercolor={1 1 1} % set to white
} 

\lstset{ %
  breaklines=true,                % sets automatic line breaking
  basicstyle=\footnotesize,           % the size of the fonts that are used for the code
  numbers=left,                   % where to put the line-numbers
  stepnumber=1,                   % the step between two line-numbers. 
  numbersep=7.5pt,                  % how far the line-numbers are from the code
  frame=single,                   % adds a frame around the code
}

%---------------------------------------------------------------------------

\author{Marek Cabaj, Radosław Gabiga,\\Andrzej Kiełtyka}
\titlePL{Aplikacja webowa}
\thesistypePL{Sesje RPG}
\date{2012}
\departmentPL{Informatyka Stosowana}
\facultyPL{Wydział Elektrotechniki, Automatyki, Informatyki i Elektroniki}
\setlength{\cftsecnumwidth}{10mm}

%---------------------------------------------------------------------------

\begin{document}

\titlepages

\tableofcontents
\clearpage

\chapter{Projekt konceptualny}
\label{cha:konceptualny}

\section{Sformułowanie zadania projektowego}
\label{sec:sforzadproj}

Temat niniejszego projektu poświęcony jest zagadnieniu prowadzenia sesji role-playing games zwanych także grami wyobraźni. Naszym celem jest projekt oraz implementacja serwisu internetowego skupiającego miłośników tego typu rozrywki w jednym miejscu. Miałby on m.in. pośredniczyć w organizowaniu sesji zarówno w formie spotkań on-line jak i off-line, pomagać graczom w tworzeniu kart postaci czy też pozwalać przechowywać scenariusze oraz inne dane dotyczące sesji. Dodatkowo miałby on do pewnego stopnia charakter portalu społecznościowego - pozwalałby użytkownikom dzielić się scenariuszami, postaciami, oceniać osoby prowadzące sesje oraz biorące w nich udział itp.

%%%%%%%%%%%%%%%%%%%%%%%%%%%%%%%%%%%%%%%%

\section{Analiza stanu wyjściowego}
\label{sec:stanwyjsciowy}

Poszukiwania przeprowadzone w Internecie nie dały prawie żadnych rezultatów. Nie istnieje żaden serwis, który obejmowałby tematykę sesji RPG w tak szerokim zakresie. Istnieją jednak pewne rozwiązania cząstkowe. Odnaleźliśmy fora dyskusyjne, gdzie ludzie starają się prowadzić sesje, pisząc posty w wyznaczonej kolejności bądź też przykłady oprogramowania typu stand-alone, które pozwala tworzyć karty postaci i dedykowane jest pojedynczym systemom, np. Dungeons \& Dragons. Ponadto pewną ciekawostką jest istnienie interaktywnych dokumentów PDF z kartami postaci.

%%%%%%%%%%%%%%%%%%%%%%%%%%%%%%%%%%%%%%%%

\section{Analiza wymagań użytkownika}
\label{sec:wymagania}

Poniżej znajduje się wstępny zakres funkcjonalności, przy czym priorytet każdego punktu został określony wg metody MoSCoW.

\renewcommand{\labelitemi}{$\bullet$}
\renewcommand{\labelitemii}{$\circ$}

\begin{itemize}
\item rejestracja użytkowników (M)
	\begin{itemize}
	\item aktywacja konta po kliknięciu w link z powiadomienia na mailu (C)
	\item ręczna aktywacja konta przez administrację (C)
	\end{itemize}
\item logowanie użytkowników (M)
\item resetowanie haseł użytkowników (wiadomość z wygenerowanym losowym hasłem na mail) (M)
\item banowanie użytkowników przez administrację (S)
\item wspomaganie organizowania sesji RPG (M)
	\begin{itemize}
	\item tworzenie / edycja / usuwanie / odpowiadanie na ogłoszenia (M)
		\begin{itemize}
		\item „poszukuję graczy” (M)
		\item „poszukuję mistrza gry” (M)
		\end{itemize}
	\item lista ogłoszeń (M)
	\item filtrowanie listy ogłoszeń (M)
	\end{itemize}
\item sesje RPG jako spotkania on-line (M)
	\begin{itemize}
	\item czat (M)
		\begin{itemize}
		\item okno sesji (M)
			\begin{itemize}
			\item lista uczestników (M)
			\item informacje o sesji (M)
			\item informacje o uczestnikach (M)
			\item rzuty kostkami wielościennymi (M)
			\end{itemize}
		\item możliwość rysowania w trakcie sesji (C)
		\end{itemize}
	\end{itemize}
\item tworzenie kart postaci dla wybranego systemu(-ów) (S)
\item charakter społecznościowy (S)
	\begin{itemize}
	\item przesyłanie prywatnych wiadomości pomiędzy użytkownikami (S)
	\item ocenianie mistrzów gry (S)
	\item ocenianie graczy (S)
	\item ustawienia prywatności / opcje udostępniania (S)
	\end{itemize}
\item baza systemów RPG (C)
\item eksport kart postaci oraz innych danych do wybranego formatu, np. pdf (C)
\item wersja mobilna serwisu (C)
\item generatory imion (C)
\item moduł sklepu pozwalający na zakup podręczników oraz innych akcesoriów związanych z sesjami RPG (W)
\end{itemize}

%%%%%%%%%%%%%%%%%%%%%%%%%%%%%%%%%%%%%%%%

\section{Przypadki użycia}
\label{sec:przypadki}

\renewcommand{\labelenumi}{\arabic{enumi}.} 
\renewcommand{\labelenumii}{\arabic{enumi}.\arabic{enumii}.}
\renewcommand{\labelenumiii}{\arabic{enumi}.\arabic{enumii}.\arabic{enumiii}.}

\begin{enumerate}
\item Gość
	\begin{enumerate}
	\item Rejestracja konta
	\item Logowanie
	\item Przeglądanie ogłoszeń
	\item Przeglądanie profili użytkowników
	\item Przeglądanie systemów RPG
	\item Resetowanie hasła
	\end{enumerate}
\item Użytkownik
	\begin{enumerate}
	\item Zarządzanie kontem
	\item Tworzenie / Edycja / Usuwanie ogłoszeń
	\item Akceptowanie zgłoszeń graczy / MG
	\item Odpowiadanie na ogłoszenia
	\item Uczestniczenie w sesji on-line
	\item Tworzenie / Edycja / Usuwanie scenariuszy
	\item Tworzenie / Edycja / Usuwanie kart postaci
	\item Ocenianie użytkowników
	\item Przesyłanie prywatnych wiadomości
	\end{enumerate}
\item Moderator
	\begin{enumerate}
	\item Edycja / Usuwanie ocen i komentarzy
	\item Dodawanie / Edycja / Usuwanie systemów RPG
	\end{enumerate}
\item Administrator
	\begin{enumerate}
	\item Zarządzanie użytkownikami
		\begin{enumerate}
		\item Ręczna aktywacja konta użytkownika
		\item Banowanie kont
		\end{enumerate}
	\end{enumerate}
\end{enumerate}


%%%%%%%%%%%%%%%%%%%%%%%%%%%%%%%%%%%%%%%%

\section{Wybrane scenariusze użycia}
\label{sec:scenariusze}

Lorem ipsum.

%%%%%%%%%%%%%%%%%%%%%%%%%%%%%%%%%%%%%%%%

\section{Identyfikacja funkcji}
\label{sec:idfun}

Główną funkcją realizowaną przez bazę danych jest przechowywanie informacji o:
	\begin{itemize}
	\item kontach użytkowników wraz z dodatkowymi informacjami (ustawieniami, ocenami, wiadomościami prywatnymi),
	\item ogłoszeniach dotyczących sesji wraz z powiązaniami do kont,
	\item postaciach graczy wraz z powiązaniami do kont,
	\item scenariuszach wraz z powiązaniami do kont,
	\item systemach RPG.
	\end{itemize}

%%%%%%%%%%%%%%%%%%%%%%%%%%%%%%%%%%%%%%%%

\clearpage

\section{FHD - diagram hierarchii funkcji}
\label{sec:FHD}

\setlength{\DTbaselineskip}{15pt}
\dirtree{%
.1 Sesje RPG.
	.2 Konta użytkowników.
		.3 Utworzenie konta.
			.4 Rejestracja.
			.4 Aktywacja.
		.3 Logowanie.
		.3 Ustawienia konta.
			.4 Ustawienia prywatności.
			.4 Ustawienia powiadomień.
			.4 Ustawienia danych w profilu.
			.4 Zmiana hasła.
		.3 Resetowanie hasła.
	.2 Ogłoszenia. 
		.3 Tworzenie ogłoszeń.
		.3 Przeglądanie ogłoszeń.
			.4 Filtrowanie ogłoszeń.
			.4 Aplikowanie do sesji.
		.3 Zarządzanie ogłoszeniami.
			.4 Edycja / Usuwanie ogłoszeń.
			.4 Akceptacja graczy.
			.4 Udział w sesji.
	.2 Scenariusze.
		.3 Tworzenie / Edycja / Usuwanie scenariuszy.
	.2 Karty postaci.
		.3 Tworzenie / Edycja / Usuwanie kart postaci.
	.2 Społeczność. 
		.3 Ocenianie użytkowników.
			.4 Wystawianie / Edycja oceny z komentarzem.
		.3 Przesyłanie prywatnych wiadomości.
	.2 Systemy RPG.
		.3 Przeglądanie bazy systemów RPG.
	.2 Administracja.
		.3 Zarządzanie użytkownikami.
			.4 Ręczna aktywacja kont.
			.4 Banowanie kont.
		.3 Zarządzanie systemami RPG.
			.4 Dodawanie / Edycja / Usuwanie systemów.
		.3 Moderacja.
			.4 Edycja / Usuwanie ocen i komentarzy.
}

%%%%%%%%%%%%%%%%%%%%%%%%%%%%%%%%%%%%%%%%

\clearpage 
\section{DFD - diagramy przepływu danych}
\label{sec:DFD}

Lorem ipsum.

%%%%%%%%%%%%%%%%%%%%%%%%%%%%%%%%%%%%%%%%

\section{Encje i atrybuty}
\label{sec:encje}

\begin{enumerate}
\item Użytkownik
	\begin{itemize}
	\item login (e-mail)
	\item hash MD5 hasła
	\item status konta
	\item poziom uprawnień
	\item ustawienia
	\item dane użytkownika
	\end{itemize}
\item Ogłoszenie
	\begin{itemize}
	\item twórca
	\item system
	\item data
	\item typ
	\item miejsce
	\item uczestnicy wraz z rolami
	\end{itemize}
\item System RPG
	\begin{itemize}
	\item nazwa
	\item opis
	\item gatunek
	\item projektant
	\item wydawca
	\item rok
	\item DTD karty postaci
	\end{itemize}
\item Karta postaci
	\begin{itemize}
	\item autor
	\item system
	\item XML
	\end{itemize}
\item Scenariusz
	\begin{itemize}
	\item autor
	\item system
	\item typ
	\item sugerowana ilość graczy
	\item treść
	\end{itemize}
\item Wiadomość
	\begin{itemize}
	\item nadawca
	\item adresat
	\item timestamp
	\item temat
	\item treść
	\end{itemize}
\item Komentarz
	\begin{itemize}
	\item komentujący
	\item komentowany
	\item ocena
	\item komentarz
	\item timestamp
	\end{itemize}
\end{enumerate}

%%%%%%%%%%%%%%%%%%%%%%%%%%%%%%%%%%%%%%%%

\section{ERD - diagramy związków encji}
\label{sec:ERD}

Lorem ipsum.

%%%%%%%%%%%%%%%%%%%%%%%%%%%%%%%%%%%%%%%%

\section{STD - diagramy przejść pomiędzy stanami}
\label{sec:STD}

Lorem ipsum.
\chapter{Projekt logiczny}
\label{cha:logiczny}

\section{Projekt bazy danych (postgreSQL)}
\label{sec:bazadanych}

\lstinputlisting[language=SQL]{./code/create_db.sql}

%%%%%%%%%%%%%%%%%%%%%%%%%%%%%%%%%%%%%%%%

\section{Słownik danych}
\label{sec:slownik}

\begin{itemize}
\item users - tabela zawierająca użytkowników serwisu, ich dane oraz ustawienia
	\begin{itemize}
	\item login - nazwa, za pomocą której użytkownik loguje się do serwisu, jego adres e-mail
	\item pass\_md5 - hash MD5 hasła użytkownika
	\item state - stan konta: aktywne, nieaktywne, zbanowane
	\item level - poziom uprawnień konta: użytkownik, moderator, administrator
	\item nickname - pseudonim, pod jakim użytkownik jest widziany przez innych
	\item location - miejsce pochodzenia użytkownika
	\item birthday - data urodzin użytkownika
	\item homepage - strona domowa użytkownika
	\item show\_chars - flaga określająca, czy użytkownik chce udostępniać innym swoje karty postaci
	\item show\_scenarios - flaga określająca, czy użytkownik chce udostępniać innym swoje scenariusze
	\item comment\_notify - flaga określająca, czy użytkownik chce otrzymywać drogą mailową powiadomienia o nowych komentarzach
	\item session\_notify - flaga określająca, czy użytkownik chce otrzymywać drogą mailową powiadomienia o osobach chcących dołączyć do jego sesji
	\item message\_notify - flaga określająca, czy użytkownik chce otrzymywać drogą mailową powiadomienia o otrzymanych nowych prywatnych wiadomościach
	\end{itemize}
\item messages - tabela z prywatnymi wiadomościami użytkowników
	\begin{itemize}
	\item id - identyfikator wiadomości
	\item addressee - adresat wiadomości
	\item sender - nadawca wiadomości
	\item time\_stamp - czas wysłania wiadomości
	\item topic - temat wiadomości
	\item content - treść wiadomości
	\item was\_read - flaga określająca, czy odbiorca przeczytał wiadomość
	\end{itemize}
\item comments - tabela z ocenami użytkowników
	\begin{itemize}
	\item id - identyfikator komentarza
	\item commentator - osoba komentująca
	\item commentee - osoba komentowana
	\item grade - ocena
	\item comment - komentarz
	\item time\_stamp - czas wystawienia komentarza
	\end{itemize}
\item rpg\_systems - tabela z systemami RPG
	\begin{itemize}
	\item id - identyfikator systemu
	\item name - nazwa systemu
	\item description - opis systemu
	\item genre - gatunek
	\item designer - projektant
	\item publisher - wydawca
	\item year - rok wydania
	\item char\_sheet\_dtd - szablon karty postaci dla systemu
	\end{itemize}
\item scenarios - tabela ze scenariuszami sesji RPG
	\begin{itemize}
	\item id - identyfikator scenariusza
	\item owner - twórca scenariusza
	\item system - system, dla którego przeznaczony jest scenariusz
	\item type - rodzaj scenariusza: kampania, część kampanii, one-shot
	\item players\_count - sugerowana ilość graczy mających brać udział w sesji
	\item content - treść scenariusza
	\end{itemize}
\item char\_sheets - tabela z kartami postaci
	\begin{itemize}
	\item id - identyfikator karty postaci
	\item owner - twórca karty postaci
	\item system - system, w ramach którego powstała karta postaci
	\item xml\_data - treść karty postaci
	\end{itemize}
\clearpage
\item sessions - tabela z ogłoszeniami dotyczącymi sesji RPG
	\begin{itemize}
	\item id - identyfikator ogłoszenia
	\item system - system, w ramach którego użytkownik chce zagrać
	\item owner - twórca ogłoszenia
	\item created - czas utworzenia ogłoszenia
	\item time\_stamp - data i czas, kiedy użytkownik chce rozegrać sesję
	\item type - rodzaj sesji: on-line, offline
	\item location - miejsce, w którym użytkownik chce rozegrać sesję
	\end{itemize}
\item participants - tabela z osobami biorącymi udział w sesji
	\begin{itemize}
	\item session - relacja do tabeli session
	\item user - relacja do tabeli users
	\item role - rola użytkownika: Mistrz Gry, Gracz
	\item state - stan: oczekujący, zaakceptowany
	\end{itemize}
\end{itemize}

%%%%%%%%%%%%%%%%%%%%%%%%%%%%%%%%%%%%%%%%

\section{Analiza zależności funkcyjnych i normalizacja tabel}
\label{sec:normalizacja}

Baza danych jest w pierwszej postaci normalnej, ponieważ każda tabela spełnia następujące warunki:
\begin{itemize}
\item opisuje jeden obiekt,
\item wartości atrybutów są elementarne (atomowe, niepodzielne) - każda kolumna jest wartością skalarną (atomową), a nie macierzą lub listą czy też czymkolwiek, co posiada własną strukturę,
\item nie zawiera kolekcji (powtarzających się grup informacji),
\item posiada klucz główny,
\item kolejność wierszy może być dowolna (znaczenie danych nie zależy od kolejności wierszy).
\end{itemize}

Baza danych jest w drugiej postaci normalnej, ponieważ jest w pierwszej postaci normalnej i każda kolumna zależy funkcyjnie od całego klucza głównego (a nie np. od części klucza). Jest także w trzeciej postaci normalnej, ponieważ spełnia warunki 2NF oraz wszystkie pola niebędące polami klucza głównego są od niego zależne bezpośrednio.

%%%%%%%%%%%%%%%%%%%%%%%%%%%%%%%%%%%%%%%%

\section{Projektowanie operacji na danych}
\label{sec:operacje}

Lorem ipsum.
\chapter{Raport końcowy}
\label{cha:raport}

\section{Wykorzystane technologie}
\label{sec:technologie}

% TODO grails, vaadin, posgreSQL - opisy, uzasadnienie?

%%%%%%%%%%%%%%%%%%%%%%%%%%%%%%%%%%%%%%%%

\section{Implementacja bazy danych}
\label{sec:impldb}

% TODO o Grails + GORM + ??? że cuda się dzieją same i działa? czy nie?

%%%%%%%%%%%%%%%%%%%%%%%%%%%%%%%%%%%%%%%%

\section{Przykładowe interfejsy użytkownika}
\label{sec:interfejsy}

% TODO screeny z jakimiś opisami, ogólny wygląd strony?
\subsection{Logowanie}
\label{sec:login}

\subsection{Strona startowa}
\label{sec:start_page}

\subsection{Lista systemów}
\label{sec:systems}

\subsection{Ogłoszenia}
\label{sec:sessions}
%join
%waiting for accepation
%create

\subsection{Lista użytkowników}
\label{sec:users_detail}

\subsection{Strona profilowa}
\label{sec:my_page}
%mypage/sessions

\subsection{FAQ}
\label{sec:faq}

\subsection{Edytowanie}
\label{sec:edit}

\subsection{Chat}
\label{sec:chat}

%obrazek chatu zrobić

%%%%%%%%%%%%%%%%%%%%%%%%%%%%%%%%%%%%%%%%

\section{Różnice pomiędzy projektem oraz implementacją}
\label{sec:roznice}

% TODO wszelkie różnice, zmiany

%%%%%%%%%%%%%%%%%%%%%%%%%%%%%%%%%%%%%%%%

\section{Rozwijanie i modyfikowanie aplikacji}
\label{sec:rozwoj}

% TODO co można oddać, rozwinąć, zmienić

%%%%%%%%%%%%%%%%%%%%%%%%%%%%%%%%%%%%%%%%

\section{Wnioski}
\label{sec:doswiadczenia}

% TODO wnioski

% TODO załączniki ?
% - Instrukcja użytkownika
%     jak postawić aplikację na serwerze, wymagane rzeczy itp.?
% - Dokumentacja techniczna
%     tu jest w ogóle coś do pisania? bo chyba nie będziemy pisać, jak korzystać z Gita i kodzić w dowolnym IDE

\addcontentsline{toc}{chapter}{Bibliografia}
\begin{thebibliography}{99}

\bibitem{bib-grails-doc}
  Rocher G., Ledbrook P., Palmer M., Brown J., Daley L., Beckwith B.
  \emph{The Grails Framework - Reference Documentation, 
  Version: 2.0.4}.
  http://grails.org/doc/latest/
  \newblock (dostęp: 20.06.2012)

\bibitem{bib-grails-api}
  \emph{The Grails Framework - API, 
  Version: 2.0.4}.
  http://grails.org/doc/latest/api/
  \newblock (dostęp: 20.06.2012)

\bibitem{bib-vaadin-doc}
  Grönroos M.
  \emph{Book of Vaadin}.
  https://vaadin.com/book
  \newblock (dostęp: 20.06.2012)

\bibitem{bib-vaadin-doc}
  \emph{Vaadin - API}.
  https://vaadin.com/api/
  \newblock (dostęp: 20.06.2012)

\bibitem{bib-postgresql-doc}
  \emph{PostgreSQL 9.1.4 Documentation}.
  http://www.postgresql.org/docs/9.1/interactive/index.html
  \newblock (dostęp: 20.06.2012)

\end{thebibliography}


\end{document}
