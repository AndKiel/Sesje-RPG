\chapter{Projekt konceptualny}
\label{cha:konceptualny}

\section{Sformułowanie zadania projektowego}
\label{sec:sforzadproj}

Temat niniejszego projektu poświęcony jest zagadnieniu prowadzenia sesji role-playing games zwanych także grami wyobraźni. Naszym celem jest projekt oraz implementacja serwisu internetowego skupiającego miłośników tego typu rozrywki w jednym miejscu. Miałby on m.in. pośredniczyć w organizowaniu sesji zarówno w formie spotkań on-line jak i off-line, pomagać graczom w tworzeniu kart postaci czy też pozwalać przechowywać scenariusze oraz inne dane dotyczące sesji. Dodatkowo miałby on do pewnego stopnia charakter portalu społecznościowego - pozwalałby użytkownikom dzielić się scenariuszami, postaciami, oceniać osoby prowadzące sesje oraz biorące w nich udział itp.

%%%%%%%%%%%%%%%%%%%%%%%%%%%%%%%%%%%%%%%%

\section{Analiza stanu wyjściowego}
\label{sec:stanwyjsciowy}

Poszukiwania przeprowadzone w Internecie nie dały prawie żadnych rezultatów. Nie istnieje żaden serwis, który obejmowałby tematykę sesji RPG w tak szerokim zakresie. Istnieją jednak pewne rozwiązania cząstkowe. Odnaleźliśmy fora dyskusyjne, gdzie ludzie starają się prowadzić sesje, pisząc posty w wyznaczonej kolejności bądź też przykłady oprogramowania typu stand-alone, które pozwala tworzyć karty postaci i dedykowane jest pojedynczym systemom, np. Dungeons \& Dragons. Ponadto pewną ciekawostką jest istnienie interaktywnych dokumentów PDF z kartami postaci.

%%%%%%%%%%%%%%%%%%%%%%%%%%%%%%%%%%%%%%%%

\section{Analiza wymagań użytkownika}
\label{sec:wymagania}

Poniżej znajduje się wstępny zakres funkcjonalności, przy czym priorytet każdego punktu został określony wg metody MoSCoW.

\renewcommand{\labelitemi}{$\bullet$}
\renewcommand{\labelitemii}{$\circ$}

\begin{itemize}
\item rejestracja użytkowników (M)
	\begin{itemize}
	\item aktywacja konta po kliknięciu w link z powiadomienia na mailu (C)
	\item ręczna aktywacja konta przez administrację (C)
	\end{itemize}
\item logowanie użytkowników (M)
\item resetowanie haseł użytkowników (wiadomość z wygenerowanym losowym hasłem na mail) (M)
\item banowanie użytkowników przez administrację (S)
\item wspomaganie organizowania sesji RPG (M)
	\begin{itemize}
	\item tworzenie / edycja / usuwanie / odpowiadanie na ogłoszenia (M)
		\begin{itemize}
		\item „poszukuję graczy” (M)
		\item „poszukuję mistrza gry” (M)
		\end{itemize}
	\item lista ogłoszeń (M)
	\item filtrowanie listy ogłoszeń (M)
	\end{itemize}
\item sesje RPG jako spotkania on-line (M)
	\begin{itemize}
	\item czat (M)
		\begin{itemize}
		\item okno sesji (M)
			\begin{itemize}
			\item lista uczestników (M)
			\item informacje o sesji (M)
			\item informacje o uczestnikach (M)
			\item rzuty kostkami wielościennymi (M)
			\end{itemize}
		\item możliwość rysowania w trakcie sesji (C)
		\end{itemize}
	\end{itemize}
\item tworzenie kart postaci dla wybranego systemu(-ów) (S)
\item charakter społecznościowy (S)
	\begin{itemize}
	\item przesyłanie prywatnych wiadomości pomiędzy użytkownikami (S)
	\item ocenianie mistrzów gry (S)
	\item ocenianie graczy (S)
	\item ustawienia prywatności / opcje udostępniania (S)
	\end{itemize}
\item baza systemów RPG (C)
\item eksport kart postaci oraz innych danych do wybranego formatu, np. pdf (C)
\item wersja mobilna serwisu (C)
\item generatory imion (C)
\item moduł sklepu pozwalający na zakup podręczników oraz innych akcesoriów związanych z sesjami RPG (W)
\end{itemize}

%%%%%%%%%%%%%%%%%%%%%%%%%%%%%%%%%%%%%%%%

\section{Scenariusze użycia}
\label{sec:scenariusze}

\renewcommand{\labelenumi}{\arabic{enumi}.} 
\renewcommand{\labelenumii}{\arabic{enumi}.\arabic{enumii}.}
\renewcommand{\labelenumiii}{\arabic{enumi}.\arabic{enumii}.\arabic{enumiii}.}

\begin{enumerate}
\item Gość
	\begin{enumerate}
	\item Wyszukiwanie gier
		\begin{enumerate}
		\item Pełna lista ogłoszeń
		\item Filtrowanie listy
		\end{enumerate}
	\item Rejestracja konta
		\begin{enumerate}
		\item Propozycja loginu
		\item Ustalenie hasła
		\item Podanie adresu email
		\item Dodatkowe dane
		\end{enumerate}
	\item Przeglądanie profili użytkowników
		\begin{enumerate}
		\item Przykładowe scenariusze
		\item Opinie użytkowników / ocena
		\item Ilość rozegranych gier
		\end{enumerate}
	\end{enumerate}
\item Użytkownik
	\begin{enumerate}
	\item Zarządzanie kontem
		\begin{enumerate}
		\item Resetowanie hasła
		\item Zmiana hasła
		\item Zmiana adresu email
		\item Zmiana pozostałych danych
		\end{enumerate}
	\item Tworzenie ogłoszeń
		\begin{enumerate}
		\item Poszukiwanie graczy (jako Mistrz Gry)
		\item Poszukiwanie Mistrza Gry (jako gracz)
		\end{enumerate}
	\item Edycja ogłoszeń
		\begin{enumerate}
		\item Modyfikacja treści
		\item Akceptowanie zgłoszeń graczy / MG
		\end{enumerate}
	\item Usuwanie ogłoszeń
	\item Odpowiadanie na ogłoszenia
		\begin{enumerate}
		\item Aplikowanie na pozycję MG
		\item Aplikowanie jako zwykły gracz
		\end{enumerate}
	\item Ocenianie użytkowników
		\begin{enumerate}
		\item Wystawianie oceny (punktowej)
		\item Wystawianie komentarza
		\end{enumerate}
	\item Sesje on-line
		\begin{enumerate}
		\item Zapraszanie graczy
		\item Wyrzucanie graczy
		\item Dołączanie do gry
		\end{enumerate}
	\item Tworzenie kart postaci
		\begin{enumerate}
		\item Wybór systemu
		\item Wypełnienie formularza
		\end{enumerate}
	\end{enumerate}

\item Administrator
	\begin{enumerate}
	\item Zarządzanie użytkownikami
		\begin{enumerate}
		\item Ręczna aktywacja konta użytkownika
		\item Usuwanie / banowanie kont
		\item Usuwanie komentarzy kont
		\end{enumerate}
	\item Usuwanie ogłoszeń
	\end{enumerate}
\end{enumerate}


%%%%%%%%%%%%%%%%%%%%%%%%%%%%%%%%%%%%%%%%

\section{Identyfikacja funkcji}
\label{sec:idfun}

Lorem ipsum.

%%%%%%%%%%%%%%%%%%%%%%%%%%%%%%%%%%%%%%%%

\section{FHD - diagramy hierarchii funkcji}
\label{sec:FHD}

Lorem ipsum.

%%%%%%%%%%%%%%%%%%%%%%%%%%%%%%%%%%%%%%%%

\section{DFD - diagramy przepływu danych}
\label{sec:DFD}

Lorem ipsum.

%%%%%%%%%%%%%%%%%%%%%%%%%%%%%%%%%%%%%%%%

\section{Encje i atrybuty}
\label{sec:encje}

Lorem ipsum.

%%%%%%%%%%%%%%%%%%%%%%%%%%%%%%%%%%%%%%%%

\section{ERD - diagramy związków encji}
\label{sec:ERD}

Lorem ipsum.

%%%%%%%%%%%%%%%%%%%%%%%%%%%%%%%%%%%%%%%%

\section{STD - diagramy przejść pomiędzy stanami}
\label{sec:STD}

Lorem ipsum.