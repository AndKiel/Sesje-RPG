\chapter{Projekt logiczny}
\label{cha:logiczny}

\section{Projekt bazy danych (postgreSQL)}
\label{sec:bazadanych}

\lstinputlisting[language=SQL]{./code/create_db.sql}

%%%%%%%%%%%%%%%%%%%%%%%%%%%%%%%%%%%%%%%%

\section{Słownik danych}
\label{sec:slownik}

\begin{itemize}
\item users - tabela zawierająca użytkowników serwisu
	\begin{itemize}
	\item login - nazwa, za pomocą której użytkownik loguje się do serwisu, jego adres e-mail
	\item pass\_md5 - hash MD5 hasła użytkownika
	\item state - stan konta: aktywne, nieaktywne, zbanowane
	\item level - poziom uprawnień konta: użytkownik, moderator, administrator
	\end{itemize}
\item users\_data - tabela z danymi użytkowników
	\begin{itemize}
	\item login - relacja do tabeli users
	\item nickname - pseudonim, pod jakim użytkownik jest widziany przez innych
	\item location - miejsce pochodzenia użytkownika
	\item birthday - data urodzin użytkownika
	\item homepage - strona domowa użytkownika
	\end{itemize}
\item users\_settings - tabela z ustawieniami użytkowników
	\begin{itemize}
	\item login - relacja do tabeli users
	\item show\_chars - flaga określająca, czy użytkownik chce udostępniać innym swoje karty postaci
	\item show\_scenarios - flaga określająca, czy użytkownik chce udostępniać innym swoje scenariusze
	\item comment\_notify - flaga określająca, czy użytkownik chce otrzymywać drogą mailową powiadomienia o nowych komentarzach
	\item session\_notify - flaga określająca, czy użytkownik chce otrzymywać drogą mailową powiadomienia o osobach chcących dołączyć do jego sesji
	\item message\_notify - flaga określająca, czy użytkownik chce otrzymywać drogą mailową powiadomienia o otrzymanych nowych prywatnych wiadomościach
	\end{itemize}
\item messages - tabela z prywatnymi wiadomościami użytkowników
	\begin{itemize}
	\item id - identyfikator wiadomości
	\item addressee - adresat wiadomości
	\item sender - nadawca wiadomości
	\item time\_stamp - czas wysłania wiadomości
	\item topic - temat wiadomości
	\item content - treść wiadomości
	\item was\_read - flaga określająca, czy odbiorca przeczytał wiadomość
	\end{itemize}
\item comments - tabela z ocenami użytkowników
	\begin{itemize}
	\item id - identyfikator komentarza
	\item commentator - osoba komentująca
	\item commentee - osoba komentowana
	\item grade - ocena
	\item comment - komentarz
	\item time\_stamp - czas wystawienia komentarza
	\end{itemize}
\item rpg\_systems - tabela z systemami RPG
	\begin{itemize}
	\item id - identyfikator systemu
	\item name - nazwa systemu
	\item description - opis systemu
	\item genre - gatunek
	\item designer - projektant
	\item publisher - wydawca
	\item year - rok wydania
	\item char\_sheet\_dtd - szablon karty postaci dla systemu
	\end{itemize}
\item scenarios - tabela ze scenariuszami sesji RPG
	\begin{itemize}
	\item id - identyfikator scenariusza
	\item owner - twórca scenariusza
	\item system - system, dla którego przeznaczony jest scenariusz
	\item type - rodzaj scenariusza: kampania, część kampanii, one-shot
	\item players\_count - sugerowana ilość graczy mających brać udział w sesji
	\item content - treść scenariusza
	\end{itemize}
\item char\_sheets - tabela z kartami postaci
	\begin{itemize}
	\item id - identyfikator karty postaci
	\item owner - twórca karty postaci
	\item system - system, w ramach którego powstała karta postaci
	\item xml\_data - treść karty postaci
	\end{itemize}
\item sessions - tabela z ogłoszeniami dotyczącymi sesji RPG
	\begin{itemize}
	\item id - identyfikator ogłoszenia
	\item system - system, w ramach którego użytkownik chce zagrać
	\item owner - twórca ogłoszenia
	\item time\_stamp - data i czas, kiedy użytkownik chce rozegrać sesję
	\item type - rodzaj sesji: on-line, offline
	\item location - miejsce, w którym użytkownik chce rozegrać sesję
	\end{itemize}
\item participants - tabela z osobami biorącymi udział w sesji
	\begin{itemize}
	\item session - relacja do tabeli session
	\item login - relacja do tabeli users
	\item role - rola użytkownika: Mistrz Gry, Gracz
	\item state - stan: oczekujący, zaakceptowany
	\end{itemize}
\end{itemize}

%%%%%%%%%%%%%%%%%%%%%%%%%%%%%%%%%%%%%%%%

\section{Analiza zależności funkcyjnych i normalizacja tabel}
\label{sec:normalizacja}

Lorem ipsum.

%%%%%%%%%%%%%%%%%%%%%%%%%%%%%%%%%%%%%%%%

\section{Denormalizacja struktury danych}
\label{sec:denormalizacja}

Lorem ipsum.

%%%%%%%%%%%%%%%%%%%%%%%%%%%%%%%%%%%%%%%%

\section{Projektowanie operacji na danych}
\label{sec:operacje}

Lorem ipsum.