\chapter{Projekt logiczny}
\label{cha:logiczny}

\section{Projekt bazy danych (postgreSQL)}
\label{sec:bazadanych}

\lstinputlisting[language=SQL]{./code/create_db.sql}

%%%%%%%%%%%%%%%%%%%%%%%%%%%%%%%%%%%%%%%%

\section{Słownik danych}
\label{sec:slownik}

\begin{itemize}
\item users - tabela zawierająca użytkowników serwisu, ich dane oraz ustawienia
	\begin{itemize}
	\item login - nazwa, za pomocą której użytkownik loguje się do serwisu, jego adres e-mail
	\item pass\_md5 - hash MD5 hasła użytkownika
	\item state - stan konta: aktywne, nieaktywne, zbanowane
	\item level - poziom uprawnień konta: użytkownik, moderator, administrator
	\item nickname - pseudonim, pod jakim użytkownik jest widziany przez innych
	\item location - miejsce pochodzenia użytkownika
	\item birthday - data urodzin użytkownika
	\item homepage - strona domowa użytkownika
	\item show\_chars - flaga określająca, czy użytkownik chce udostępniać innym swoje karty postaci
	\item show\_scenarios - flaga określająca, czy użytkownik chce udostępniać innym swoje scenariusze
	\item comment\_notify - flaga określająca, czy użytkownik chce otrzymywać drogą mailową powiadomienia o nowych komentarzach
	\item session\_notify - flaga określająca, czy użytkownik chce otrzymywać drogą mailową powiadomienia o osobach chcących dołączyć do jego sesji
	\item message\_notify - flaga określająca, czy użytkownik chce otrzymywać drogą mailową powiadomienia o otrzymanych nowych prywatnych wiadomościach
	\end{itemize}
\item messages - tabela z prywatnymi wiadomościami użytkowników
	\begin{itemize}
	\item id - identyfikator wiadomości
	\item addressee - adresat wiadomości
	\item sender - nadawca wiadomości
	\item time\_stamp - czas wysłania wiadomości
	\item topic - temat wiadomości
	\item content - treść wiadomości
	\item was\_read - flaga określająca, czy odbiorca przeczytał wiadomość
	\end{itemize}
\item comments - tabela z ocenami użytkowników
	\begin{itemize}
	\item id - identyfikator komentarza
	\item commentator - osoba komentująca
	\item commentee - osoba komentowana
	\item grade - ocena
	\item comment - komentarz
	\item time\_stamp - czas wystawienia komentarza
	\end{itemize}
\item rpg\_systems - tabela z systemami RPG
	\begin{itemize}
	\item id - identyfikator systemu
	\item name - nazwa systemu
	\item description - opis systemu
	\item genre - gatunek
	\item designer - projektant
	\item publisher - wydawca
	\item year - rok wydania
	\item char\_sheet\_dtd - szablon karty postaci dla systemu
	\end{itemize}
\item scenarios - tabela ze scenariuszami sesji RPG
	\begin{itemize}
	\item id - identyfikator scenariusza
	\item owner - twórca scenariusza
	\item system - system, dla którego przeznaczony jest scenariusz
	\item type - rodzaj scenariusza: kampania, część kampanii, one-shot
	\item players\_count - sugerowana ilość graczy mających brać udział w sesji
	\item content - treść scenariusza
	\end{itemize}
\item char\_sheets - tabela z kartami postaci
	\begin{itemize}
	\item id - identyfikator karty postaci
	\item owner - twórca karty postaci
	\item system - system, w ramach którego powstała karta postaci
	\item xml\_data - treść karty postaci
	\end{itemize}
\clearpage
\item sessions - tabela z ogłoszeniami dotyczącymi sesji RPG
	\begin{itemize}
	\item id - identyfikator ogłoszenia
	\item system - system, w ramach którego użytkownik chce zagrać
	\item owner - twórca ogłoszenia
	\item created - czas utworzenia ogłoszenia
	\item time\_stamp - data i czas, kiedy użytkownik chce rozegrać sesję
	\item type - rodzaj sesji: on-line, offline
	\item location - miejsce, w którym użytkownik chce rozegrać sesję
	\end{itemize}
\item participants - tabela z osobami biorącymi udział w sesji
	\begin{itemize}
	\item session - relacja do tabeli session
	\item user - relacja do tabeli users
	\item role - rola użytkownika: Mistrz Gry, Gracz
	\item state - stan: oczekujący, zaakceptowany
	\end{itemize}
\end{itemize}

%%%%%%%%%%%%%%%%%%%%%%%%%%%%%%%%%%%%%%%%

\section{Analiza zależności funkcyjnych i normalizacja tabel}
\label{sec:normalizacja}

\hspace{15pt}Baza danych jest w pierwszej postaci normalnej, ponieważ każda tabela spełnia następujące warunki:
\begin{itemize}
\item opisuje jeden obiekt,
\item wartości atrybutów są elementarne (atomowe, niepodzielne) - każda kolumna jest wartością skalarną (atomową), a nie macierzą lub listą czy też czymkolwiek, co posiada własną strukturę,
\item nie zawiera kolekcji (powtarzających się grup informacji),
\item posiada klucz główny,
\item kolejność wierszy może być dowolna (znaczenie danych nie zależy od kolejności wierszy).
\end{itemize}

\hspace{15pt}Baza danych jest w drugiej postaci normalnej, ponieważ jest w pierwszej postaci normalnej i każda kolumna zależy funkcyjnie od całego klucza głównego (a nie np. od części klucza). Jest także w trzeciej postaci normalnej, ponieważ spełnia warunki 2NF oraz wszystkie pola niebędące polami klucza głównego są od niego zależne bezpośrednio.

%%%%%%%%%%%%%%%%%%%%%%%%%%%%%%%%%%%%%%%%

\clearpage
\section{Projektowanie operacji na danych}
\label{sec:operacje}

\begin{itemize}
\item rejestracja nowego użytkownika
\begin{lstlisting}[language=SQL]
INSERT INTO users (login, pass_md5, nickname) VALUES (login_value, pass_md5_value, nickname_value);
\end{lstlisting}

\item aktywacja konta użytkownika
\begin{lstlisting}[language=SQL]
UPDATE users SET state = 'A' WHERE login = login_value;
\end{lstlisting}

\item logowanie użytkownika na stronę
\begin{lstlisting}[language=SQL]
EXISTS (SELECT login, pass_md5 FROM users WHERE login = login_value AND pass_md5 = pass_md5_value);
\end{lstlisting}

\item zmiana lub resetowanie hasła użytkownika
\begin{lstlisting}[language=SQL]
UPDATE users SET pass_md5 = new_pass_md5 WHERE login = login_value;
\end{lstlisting}

\item edycja danych konta użytkownika
\begin{lstlisting}[language=SQL]
UPDATE users SET location = new_location, birthday = new_birthday, homepage = new_homapage, show_chars = tru_fal_value1, show_scenarios = tru_fal_value2, comment_notify = tru_fal_value3, session_notify = tru_fal_value4, message_notify = tru_fal_value5 WHERE login = login_value;
\end{lstlisting}

\item pobieranie listy użytkowników
\begin{lstlisting}[language=SQL]
SELECT nickname, location, birthday, homepage FROM users;
\end{lstlisting}

\item banowanie konta użytkownika
\begin{lstlisting}[language=SQL]
UPDATE users SET state = 'B' WHERE login = login_value;
\end{lstlisting}

\item pobieranie prywatnych wiadomości przez użytkownika
\begin{lstlisting}[language=SQL]
SELECT sender, time_stamp, topic, content, was_read FROM messages WHERE addressee = current_user;
\end{lstlisting}

\item wysyłanie prywatnej wiadomości użytkownikowi
\begin{lstlisting}[language=SQL]
INSERT INTO messages (addressee, sender, topic, content) VALUES (user_id, current_user, topic_value, content_value);
\end{lstlisting}

\item usuwanie prywatnej wiadomości przez użytkownika
\begin{lstlisting}[language=SQL]
DELETE FROM messages WHERE id = msg_id;
\end{lstlisting}

\item pobieranie komentarzy wystawionych przez użytkownika
\begin{lstlisting}[language=SQL]
SELECT commentee, grade, comment, time_stamp FROM comments WHERE comentator = user_id;
\end{lstlisting}

\item pobieranie komentarzy dotyczących użytkownika
\begin{lstlisting}[language=SQL]
SELECT comentator, grade, comment, time_stamp FROM comments WHERE commentee = user_id;
\end{lstlisting}

\item wystawianie komentarza
\begin{lstlisting}[language=SQL]
INSET INTO comments (comentator, commentee, grade, comment) VALUES (current_user, user_id, grade_value, comment_value);
\end{lstlisting}

\item edycja komentarza
\begin{lstlisting}[language=SQL]
UPDATE comments SET grade = new_grade, comment = new_comment WHERE id = comment_id;
\end{lstlisting}

\item usunięcie komentarza
\begin{lstlisting}[language=SQL]
DELETE FROM comments WHERE id = comment_id;
\end{lstlisting}

\item pobieranie wszystkich kart postaci danego użytkownika
\begin{lstlisting}[language=SQL]
SELECT system, xml_data FROM char_sheets WHERE owner = user_id;
\end{lstlisting}

\item tworzenie karty postaci przez użytkownika
\begin{lstlisting}[language=SQL]
INSERT INTO char_sheets (owner, system, xml_data) VALUES (current_user, system_id, xml_data_value);
\end{lstlisting}

\item edycja karty postaci przez użytkownika
\begin{lstlisting}[language=SQL]
UPDATE char_sheets SET system = new_system, xml_data = new_xml_data WHERE id = char_sheet_id;
\end{lstlisting}

\item usuwanie karty postaci przez użytkownika
\begin{lstlisting}[language=SQL]
DELETE FROM char_sheets WHERE id = char_sheet_id;
\end{lstlisting}

\item pobieranie wszystkich scenariuszy danego użytkownika
\begin{lstlisting}[language=SQL]
SELECT system, type, players_count, content FROM scenarios WHERE owner = user_id;
\end{lstlisting}

\clearpage
\item tworzenie scenariusza przez użytkownika
\begin{lstlisting}[language=SQL]
INSERT INTO scenarios (owner, system, type, players_count, content) VALUES (current_user, system_id, type_value, players_count_value, content_value);
\end{lstlisting}

\item edycja scenariusza przez użytkownika
\begin{lstlisting}[language=SQL]
UPDATE scenarios SET system = new_system, type = new_type, players_count = new_players_count, content = new_content WHERE id = scenario_id;
\end{lstlisting}

\item usuwanie scenariusza prez użytkownika
\begin{lstlisting}[language=SQL]
DELETE FROM scenarios WHERE id = scenario_id;
\end{lstlisting}

\item pobieranie listy systemów RPG
\begin{lstlisting}[language=SQL]
SELECT id, name, description, genre, designer, publisher, year FROM rpg_systems;
\end{lstlisting}

\item dodawanie systemu RPG
\begin{lstlisting}[language=SQL]
INSERT INTO rpg_systems (name, description, genre, designer, publisher, year) VALUES (name_value, description_value, genre_value, designer_value, publisher_value, year_value);
\end{lstlisting}

\item edycja systemu RPG
\begin{lstlisting}[language=SQL]
UPDATE rpg_systems SET name = new_name, description = new_description, genre = new_genre, designer = new_designer, publisher = new_publisher, year = new_year WHERE id = rpg_system_id;
\end{lstlisting}

\item usuwanie systemu RPG
\begin{lstlisting}[language=SQL]
DELETE FROM rpg_systems WHERE id = rpg_system_id;
\end{lstlisting}

\item pobieranie listy ogłoszeń (sesji)
\begin{lstlisting}[language=SQL]
SELECT id, system, owner, time_stamp, type, location FROM sessions;
\end{lstlisting}

\item pobieranie listy ogłoszeń (sesji) użytkownika
\begin{lstlisting}[language=SQL]
SELECT id, system, time_stamp, type, location FROM sessions WHERE owner = current_user;
\end{lstlisting}

\clearpage
\item tworzenie ogłoszenia (sesji)
\begin{lstlisting}[language=SQL]
INSERT INTO sessions (system, owner, time_stamp, type, location) VALUES (system_id, current_user, when, ses_type, where);
\end{lstlisting}

\item usuwanie ogłoszenia (sesji)
\begin{lstlisting}[language=SQL]
DELETE FROM sessions WHERE id = session_id;
\end{lstlisting}

\item odpowiadanie na ogłoszenie (dołączanie do sesji)
\begin{lstlisting}[language=SQL]
INSERT INTO participants (session, user, role) VALUES (session_id, current_user, session_role);
\end{lstlisting}

\item akceptacja odpowiedzi na ogłoszenie
\begin{lstlisting}[language=SQL]
UPDATE participants SET state = true WHERE session = session_id, user = user_id;
\end{lstlisting}

\item odrzucenie odpowiedzi na ogłoszenie lub rezygnacja przez odpowiadającego
\begin{lstlisting}[language=SQL]
DELETE FROM participants WHERE session = session_id, user = user_id;
\end{lstlisting}
\end{itemize}