\chapter{Raport końcowy}
\label{cha:raport}

\section{Wykorzystane technologie}
\label{sec:technologie}

% TODO grails, vaadin, posgreSQL - opisy, uzasadnienie?

%%%%%%%%%%%%%%%%%%%%%%%%%%%%%%%%%%%%%%%%

\section{Implementacja bazy danych}
\label{sec:impldb}

% TODO o Grails + GORM + ??? że cuda się dzieją same i działa? czy nie?

%%%%%%%%%%%%%%%%%%%%%%%%%%%%%%%%%%%%%%%%

\section{Przykładowe interfejsy użytkownika}
\label{sec:interfejsy}

% TODO screeny z jakimiś opisami, ogólny wygląd strony?
\subsection{Logowanie}
\label{sec:login}
Na rysunku \ref{fig:login} przedstawione zostało okno logowania. Użytkownik podaje swój login i hasło oraz wybiera, czy chce aby dane były zapamiętane w plikach cookies, co umożliwi pozostanie zalogowanym po ponownym włączeniu strony lub jej odświeżeniu. W przypadku gdy użytkownik zapomniał hasła, może nacisnąć przycisk \emph{Forgotten password? Click here} i ponownie wpisać adres e-mail, na który zarejestrowane jest konto. Dzięki temu system zresetuje hasło na losowy ciąg znaków i wyśle je do użytkownika.

\begin{figure}[h!]	
\centering
\includegraphics[width=0.5\textwidth]{./img/interfejsy/login2}
\caption{Okno logowania}
\label{fig:login}
\end{figure}

\subsection{Strona startowa}
\label{sec:start_page}

Po zalogowaniu, na stronie startowej pojawiają się informacje o nadchodzących sesjach. Na rysunku \ref{fig:start_page_joined} ciemniejszym kolorem oznaczone są sesje, do których użytkownik w pełni dołączył, jaśniejszym --- oczekujące na akceptację przez założyciela ogłoszenia. Dla zalogowanych użytkowników możliwa jest nawigacja do kolejnych podstron --- zakładek.
\begin{figure}[htb]	
\centering
\includegraphics[width=0.9\textwidth]{./img/interfejsy/start_page_joined}
\caption{Strona startowa zalogowanego użytkownika}
\label{fig:start_page_joined}
\end{figure}

\subsection{Lista systemów}
\label{sec:systems}
Wybranie zakładki \emph{Systems} powoduje przejście do listy systemów RPG zapisanych w bazie serwisu. Składa się ona z tabeli systemów z podziałem na gatunki i rok wydania oraz opisu wybranej pozycji znajdującego się poniżej. Listę można filtrować względem pierwszej litery z nazwy lub wyświetlać wszystkie. Jeśli zalogowany użytkownik jest administratorem lub moderatorem, widoczne stają się przyciski \emph{New system}, \emph{Edit} i \emph{Delete} kolejno do dodawania nowego systemu, edycji istniejącego oraz usuwania. Całość przedstawia rysunek \ref{fig:systems}.

\begin{figure}[htb]	
\centering
\includegraphics[width=0.9\textwidth]{./img/interfejsy/systems}
\caption{Lista systemów RPG}
\label{fig:systems}
\end{figure}

\subsection{Ogłoszenia}
\label{sec:sessions}
Zakładka \emph{Announcements} stanowi główny cel serwisu --- ogłoszenia o odbywających się sesjach (res. \ref{fig:sessions}).  Znajduje się tutaj tabela z kluczowymi informacjami oraz rozwinięcie dostępne po zaznaczeniu wybranej pozycji. Sesje oznaczane są takimi samymi kolorami jak na stronie startowej. Po wybraniu sesji i naciśnięciu przycisku \emph{Join} pojawia się okno wyboru pozycji w trakcie sesji (rys. \ref{fig:join_session}). Można dołączyć jako Mistrz Gry (\emph{Master}) lub jako Gracz (\emph{Player}). W przypadku, gdy dołączający użytkownik nie jest założycielem ogłoszenia, po wyborze pozycji wysyłana jest wiadomość z prośbą o akceptację do osoby odpowiedzialnej za sesję. 

\begin{figure}[htb]	
\centering
\includegraphics[width=0.9\textwidth]{./img/interfejsy/sessions}
\caption{Lista ogłoszeń}
\label{fig:sessions}
\end{figure}

\begin{figure}[htb]	
\centering
\includegraphics[width=0.3\textwidth]{./img/interfejsy/join_session}
\caption{Dołączanie do sesji}
\label{fig:join_session}
\end{figure}

Po naciśnięciu przycisku \emph{New Session} pojawia się okno tworzenia nowej sesji. Użytkownik podaje wymagane dane, wybiera czy od razu chce do niej dołączyć jako Gracz lub Mistrz Gry i dodaje ogłoszenie do bazy. Formularz ten jest przedstawiony na rysunku \ref{fig:create_session}.

\begin{figure}[htb]	
\centering
\includegraphics[width=0.7\textwidth]{./img/interfejsy/create_session}
\caption{Tworzenie nowej sesji}
\label{fig:create_session}
\end{figure}


%waiting for accepation ???


\subsection{Lista użytkowników}
\label{sec:users_detail}
Zakładka \emph{Users} (rys. \ref{fig:users_details} to lista użytkowników zamierająca pseudonim, datę dołączenia kraj pochodzenia oraz status konta (aktywne, nieaktywne). Użytkownicy mogą wyświetlić szczegóły profilu po wybraniu danej osoby i naciśnięciu \emph{Show details}. Gdy zalogowany jest administrator lub moderator widoczny jest również przycisk \emph{Deactivate account} służący do banowania użytkowników. 

\begin{figure}[htb]	
\centering
\includegraphics[width=0.9\textwidth]{./img/interfejsy/users_details}
\caption{Szczegóły profilu użytkownika}
\label{fig:users_details}
\end{figure}

\subsection{Strona profilowa}
\label{sec:my_page}
Zakładka \emph{My Page} zawiera profil zalogowanego użytkownika. Może tutaj przeglądać i edytować dane profilowe, sprawdzać wiadomości i powiadomienia systemowe, przeglądać sesje, w których uczestniczy, oczekuje na akceptacje lub które stworzył. Ponadto można tworzyć scenariusze rozgrywek jak i (w przyszłości) karty postaci. Administrator dodatkowo posiada możliwość dodawania wiadomości na stronę główną (\emph{News}) i pytań w dziale \emph{FAQ}. Przykład profilu ze stworzonymi sesjami przedstawia rysunek \ref{fig:mypage_sessions}.

\begin{figure}[htb]	
\centering
\includegraphics[width=0.9\textwidth]{./img/interfejsy/mypage_sessions}
\caption{Sesje użytkownika}
\label{fig:mypage_sessions}
\end{figure}

\subsection{FAQ}
\label{sec:faq}

\subsection{Edytowanie}
\label{sec:edit}

\subsection{Chat}
\label{sec:chat}

%obrazek chatu zrobić

%%%%%%%%%%%%%%%%%%%%%%%%%%%%%%%%%%%%%%%%

\section{Różnice pomiędzy projektem oraz implementacją}
\label{sec:roznice}

% TODO wszelkie różnice, zmiany

%%%%%%%%%%%%%%%%%%%%%%%%%%%%%%%%%%%%%%%%

\section{Rozwijanie i modyfikowanie aplikacji}
\label{sec:rozwoj}

% TODO co można oddać, rozwinąć, zmienić

%%%%%%%%%%%%%%%%%%%%%%%%%%%%%%%%%%%%%%%%

\section{Wnioski}
\label{sec:doswiadczenia}

% TODO wnioski